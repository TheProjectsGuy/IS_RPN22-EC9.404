% !TeX root = main.tex
\section{Accommodating More Waypoints}

We could accommodate more waypoints through the following methods

\begin{enumerate}
    \item Increase the value of $n$ ($n$-th degree Bernstein basis). This will allow us to introduce more waypoints while keeping the system \emph{critically constrained}.
    \item We could directly add the waypoints, with the same $n$. This will make the system \emph{over-constrained}.
\end{enumerate}

\subsection{Piece-wise Bernstein Polynomials}

However, we usually have a global path planner (like RRT) running at lower frequencies (say $10$ Hz). That path planner gives many points (consider them to be waypoints, with the exception of start and goal).

At a different frequency, and within a small neighborhood, we need to travel from one point to another (on this proposed solution by the global path planner) through waypoints. This is done through a piece-wise fitting of Bernstein Polynomials while keeping the end constraints in check (for position and velocity).

Through this method, we can theoretically accommodate all the points proposed by the global path planner into a big piece-wise Bernstein polynomial.
