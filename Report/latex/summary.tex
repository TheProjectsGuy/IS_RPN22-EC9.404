% !TeX root = main.tex
\section{Summary}

\subsection{Robotic Planning and Navigation - EC4-403}
This is a course that the independent study tried to follow. The following was completed from this course

\begin{itemize}
    \item Assignment 1: Motion planning using RRT. Official submission at \href{https://github.com/Robotics-Planning-Navigation/assignment1-over-9000/tree/dev-avneesh}{Robotics-Planning-Navigation}.
    \item Assignment 2: Bernstein polynomials. Official submission at \href{https://github.com/Robotics-Planning-Navigation/assignment-two-TheProjectsGuy}{Robotics-Planning-Navigation}.
    \item Assignment 3: Time scaling, collision cone. Official submission at \href{https://github.com/Robotics-Planning-Navigation/assignment-2b-over-9000/tree/dev-avneesh}{Robotics-Planning-Navigation}.
\end{itemize}

Additionally, lecture videos were also seen (as long as they were online and recorded). Lectures videos from \texttt{Lecture 3} through \texttt{Lecture 14} were seen for this study. The following concepts were covered through watching the recorded lectures

\begin{displayquote}
    Visibility Graph - Forward and Inverse Kinematics of differential drive robots - UAV motion control - Model Predictive Controller - Collision checking as an optimization problem - Bernstein polynomials - Time scaling - Collision cone and Velocity obstacle - Time scaled collision cone - Inverse Velocity obstacle
\end{displayquote}

\subsection{Extra Material}

The following was covered by referring to online sources \footnote{AP102 - Motion planning and Path tracking - Naveen Arulselvan}

\begin{displayquote}
    Unicycle and bicycle model - wheel encoders - creating a dashboard using Plotly - RRT - \textbf{ Dijkstra} from scratch - navigating indian cities using \textbf{OSMNX} - Trajectory generation using cubic and quintic spirals - tracking trajectories using \textbf{pure pursuit and stanley trackers} - Collision checking using circles and convex hulls (swath) - \textbf{dynamic window avoidance}
\end{displayquote}

The above topics are also included in the repository.
